\documentclass[a4paper,10pt]{article}
\usepackage[utf8]{inputenc}
\usepackage[spanish]{babel}
\usepackage{amsmath}

%opening
\title{{\bf Trabajo Práctico III:}\\ \emph{Procesamiento de imágenes III}}
\author{P. Pérez}

\footnotetext[1]{Curso: {\emph{Introducción al procesamiento de imágenes radiológicas en el ámbito clínico}}}


\begin{document}

\maketitle

\begin{abstract}
Los ejercicios comprenden el procesamiento de imágenes con derivadas y su utilización en la detección de esquinas y bordes. Los mismos deben ser entregados en formato PDF o en un notebook de Jupyter especificando código de programación utilizado, funciones implementadas y resultados obtenidos. Se aceptarán trabajos realizados en plataformas Matlab y Python.
\end{abstract}

\section*{Ejercicios}

\begin{enumerate}
 \item Calcular la derivada primera de una imagen médica.
 \item Interpretar y aplicar el método de Sobel a una imagen médica. Obtener las componentes $\nabla_{m}$ y $\nabla_{n}$ del operador gradiente.
 \item Aplicar el método de Sobel utilizando como parámetro de umbralamiento el valor medio de la imagen médica a fin de calcular y graficar la imagen binarizada resultante, la imagen de gradiente $|\vec{\nabla}|$ y la imagen de ángulo $\theta$.
 \item Aplicar el método de Prewit a una imagen médica. Comparar el resultado obtenido con el método de Sobel. Interpretar.
\end{enumerate}


\end{document}
