\documentclass[a4paper,10pt]{article}
\usepackage[utf8]{inputenc}
\usepackage[spanish]{babel}
\usepackage{amsmath}

%opening
\title{{\bf Trabajo Práctico VI:}\\ \emph{Monte Carlo}}
\author{P. Pérez}

\footnotetext[1]{Curso: {\emph{Introducción al procesamiento de imágenes radiológicas en el ámbito clínico}}}


\begin{document}

\maketitle

\begin{abstract}
Los ejercicios comprenden el capítulo de configuraciones radiológicas en simulación Monte Carlo. Los mismos deben ser entregados en formato PDF  o Jupyter Notebook especificando código de programación utilizado, funciones implementadas y resultados obtenidos. Se aceptarán trabajos realizados en plataformas Matlab y Python.
\end{abstract}

\section*{Ejercicios}

\begin{enumerate}
 \item  Realizar una simulación Monte Carlo para determinar la energía absorbida a lo largo del eje de incidencia para un haz de fotones de 10 MeV incidiendo en un fantoma de agua cúbico de lado 20 cm. El haz es perfectamente colimado. Resolver el problema usando PENELOPE y FLUKA.
 \item Repetir el ejercicio del item anterior para un haz de electrones de igual energía.
 \item Analizar y discutir los resultados obtenidos.
 \item Estudiar cómo puede simularse el problema de determinar la dosis absorbida a lo largo del eje central en un fantoma de músculo blando irradiado por un haz de 1 H de 400 MeV perfectamente colimado.
\end{enumerate}

\end{document}
