\documentclass[a4paper,10pt]{article}
\usepackage[utf8]{inputenc}
\usepackage[spanish]{babel}
\usepackage{amsmath}

%opening
\title{{\bf Trabajo Práctico IV:}\\ \emph{Procesos estocásticos}}
\author{P. Pérez}

\footnotetext[1]{Curso: {\emph{Introducción al procesamiento de imágenes radiológicas en el ámbito clínico}}}


\begin{document}

\maketitle

\begin{abstract}
Los ejercicios comprenden el capítulo de Procesos Estocásticos. Los mismos deben ser entregados en formato PDF o en un notebook de Jupyter especificando código de programación utilizado, funciones implementadas y resultados obtenidos. Se aceptarán trabajos realizados en plataformas Matlab y Python.
\end{abstract}

\section*{Ejercicios}

\begin{enumerate}
 \item  Determinar la distribución de probabilidades de las realizaciones del lanzamiento de un dado en función del muestreo (número de repeticiones de la experiencia). Graficar, determinar valor medio y varianza. Analizar los resultados.
 \item Determinar la distribución de probabilidades de las realizaciones del lanzamiento de tres dados de 10 caras cada uno en función del muestreo (número de repeticiones de la experiencia). Graficar, determinar valor medio y varianza. Analizar los resultados.
 \item Plantear un método para resolver el flujo de energía o partículas en un sistema constituido por una fuente puntual monoenergética de emisión a tasa constante y unidireccional (usualmente denominado kernel de haz) localizada en un medio homogéneo e isotrópico cuyas propiedades de absorción/scattering son conocidas.
 \item Aplicar la situación del item precedente en el caso en que el medio irradiado consiste de un slab (trozo) de material de espesor D a lo largo del eje de incidencia y de dimensiones infinitas a fines prácticos en las direcciones transversales. Asumiendo, que en primera aproximación los efectos de interacción pueden reducirses a eventos de tipo elásticos con desviaciones angulares están determinadas por la función $f(\cos{\theta})$, obtener una expresión a partir de la cual podría calcularse el flujo de energía o partículas.
 \end{enumerate}


\end{document}
