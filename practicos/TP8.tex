\documentclass[a4paper,10pt]{article}
\usepackage[utf8]{inputenc}
\usepackage[spanish]{babel}
\usepackage{amsmath}
\usepackage{graphicx}

%opening
\title{{\bf Trabajo Práctico VII:}\\ \emph{Radiodiagnóstico Metabólico}}
\author{P. Pérez}

\footnotetext[1]{Curso: {\emph{Introducción al procesamiento de imágenes radiológicas en el ámbito clínico}}}


\begin{document}

\maketitle

\begin{abstract}
Los ejercicios comprenden el capítulo de radiodiagnóstico metabólico. Los mismos deben ser entregados en formato PDF o Jupyter Notebook especificando código de programación utilizado, funciones implementadas y resultados obtenidos. Se aceptarán trabajos realizados en plataformas Matlab y Python.
\end{abstract}

\section*{Ejercicios}

\begin{enumerate}
 \item Realizar una simulación determinista de la formación de la imagen generada en un plano de detección por contaste de absorción de una esfera de radio R cuyo centro dista L del plano de detección y dentro de la misma existe un punto de emisión de fotones de 40 keV en el centro de la esfera. Estudiar el efecto de los parámetros R y L. La fuente consta de 10 mCi. Interpretar los resultados obtenidos de acuerdo con la figura \ref{fig}.
 \item  Repetir el ejercicio del item anterior utilizando el código PENELOPE. Estudiar los sistemas de colimación que se requieren. Analizar y discutir los resultados obtenidos.
 \item Realizar una simulación Monte Carlo análoga al item anterior pero clocando 3 fuentes puntales dentro de la esfera.
 \item  Implementar el método de aceptación/rechazo para modelar la emisión de una fuente de $^{99m}$Tc uniformemente distibuida en en un cubo de 1 cm de lado inmerso en una esfera 10 cm de diámetro. Realizar simulaciones determinista y Monte Carlo de lo que reprentaría el imaging con cámara gamma.
\end{enumerate}

\begin{figure}
 \centering
 \includegraphics[width=0.75\textwidth]{fig.png}
 \caption{Respuesta del sistema a una fuente puntual colocada a distintas distancias del plano del colimador (colimador de agujeros paralelos).}
 \label{fig}
\end{figure}


\end{document}
