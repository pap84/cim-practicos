\documentclass[a4paper,10pt]{article}
\usepackage[utf8]{inputenc}
\usepackage[spanish]{babel}
\usepackage{amsmath}

%opening
\title{{\bf Trabajo Práctico VII:}\\ \emph{Radiodiagnóstico Anatómico}}
\author{P. Pérez}

\footnotetext[1]{Curso: {\emph{Introducción al procesamiento de imágenes radiológicas en el ámbito clínico}}}


\begin{document}

\maketitle

\begin{abstract}
Los ejercicios comprenden el capítulo de radiodiagnóstico anatómico. Los mismos deben ser entregados en formato PDF o Jupyter Notebook especificando código de programación utilizado, funciones implementadas y resultados obtenidos. Se aceptarán trabajos realizados en plataformas Matlab y Python.
\end{abstract}

\section*{Ejercicios}

\begin{enumerate}
 \item Realizar una simulación determinista de la formación de la imagen por contaste de absorción de un cilindro de agua dede 10 cm de diámetro y altura irradiado con un haz paralelo de fotones de 50 keV y tamaño de campo de 10 cm $\times$ 10 cm. (Detección ideal). Graficar perfil central de la imagen e interpretarlo en términos de las propiedades físicas.
 \item Repetir el ejercicio del item anterior para un haz incidente de 2 canales energéticos igualmente probables de 50 keV y 30 keV. Analizar y discutir los resultados obtenidos.
 \item Realizar una simulación Monte Carlo análoga al primer item. Analizar los resultados.
 \item Simular un set up experimental típico del instrumento de laboratorio.
\end{enumerate}

\end{document}
